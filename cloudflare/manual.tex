%!TEX TS-program = xelatex
%!TEX encoding = UTF-8 Unicode

\documentclass[12pt]{article}
\usepackage{geometry}
\usepackage{float}
\usepackage{amssymb}
\usepackage{amsmath}
\usepackage{tikz}
\usepackage{minted}
\usepackage{caption}
\usepackage[font={small,it}]{caption}
\usepackage{booktabs,makecell,multirow,threeparttable}
\usepackage{graphicx}
\graphicspath{{./images/}}

\usepackage{fancyhdr}
\setlength{\headheight}{25pt}
\pagestyle{fancy}
\fancyhf{}
\fancyhead[L]{\leftmark}
\fancyhead[R]{\rightmark}
\fancyfoot[C]{\thepage}

\usepackage[utf8]{inputenc}
\usepackage{fontspec,xltxtra,xunicode}
\usepackage[T1]{fontenc}
\usepackage{biolinum}
\usepackage{libertineRoman} % or any other font package
\usepackage[
  colorlinks=true,
  linkcolor=blue,
  bookmarksnumbered=true,
  bookmarksopen=true]{hyperref}

\newcommand{\mytitle}[1]{
  {\Huge #1}\par
}

\newcommand{\mysubtitle}[1]{
  {\emph{#1}}\par
}

\definecolor{titlepagecolor}{cmyk}{1,.60,0,.40}

\DeclareFixedFont{\titlefont}{T1}{ppl}{b}{it}{1.0in}

\newminted{bash}{
  autogobble,
  breaklines,
  frame=leftline,
  framerule=1.2pt,
  framesep=1em,
  linenos,
  fontsize=\footnotesize
}

%!TEX TS-program = lualatex
%!TEX r = UTF-8 Unicode
%!TeX root = manual.tex

\newcommand\titlepagedecoration{%
\begin{tikzpicture}[remember picture,overlay,shorten >= -10pt]

\coordinate (aux1) at ([yshift=-15pt]current page.north east);
\coordinate (aux2) at ([yshift=-410pt]current page.north east);
\coordinate (aux3) at ([xshift=-4.5cm]current page.north east);
\coordinate (aux4) at ([yshift=-150pt]current page.north east);

\begin{scope}[titlepagecolor!40,line width=12pt,rounded corners=12pt]
\draw
  (aux1) -- coordinate (a)
  ++(225:5) --
  ++(-45:5.1) coordinate (b);
\draw[shorten <= -10pt]
  (aux3) --
  (a) --
  (aux1);
\draw[opacity=0.6,titlepagecolor,shorten <= -10pt]
  (b) --
  ++(225:2.2) --
  ++(-45:2.2);
\end{scope}
\draw[titlepagecolor,line width=8pt,rounded corners=8pt,shorten <= -10pt]
  (aux4) --
  ++(225:0.8) --
  ++(-45:0.8);
\begin{scope}[titlepagecolor!70,line width=6pt,rounded corners=8pt]
\draw[shorten <= -10pt]
  (aux2) --
  ++(225:3) coordinate[pos=0.45] (c) --
  ++(-45:3.1);
\draw
  (aux2) --
  (c) --
  ++(135:2.5) --
  ++(45:2.5) --
  ++(-45:2.5) coordinate[pos=0.3] (d);
\draw
  (d) -- +(45:1);
\end{scope}
\end{tikzpicture}%
}


\begin{document}

\begin{titlepage}
\begin{minipage}{0.8\linewidth}
  \vspace*{4cm}
  \begin{center}
    \noindent
    \mytitle{Results of experimenting with Brotli for dynamic web
    content}
    \rule{\linewidth}{0.2ex}\par
    \mysubtitle{Vlad Krasnov \\ typeset by Justin Zhang}
  \end{center}
\end{minipage}
\null \vfill
\hfill
\begin{minipage}{0.55\linewidth}
  \begin{flushright}
    % \printauthor
  \end{flushright}
\end{minipage}
%
% \begin{minipage}{0.02\linewidth}
%   \rule{1pt}{260pt}
% \end{minipage}
\titlepagedecoration
\end{titlepage}

\newpage
\tableofcontents
% \newpage
% \listoffigures
% \newpage
% \listoftables
% \newpage
% \listoflistings

\newpage
This document is typesetted by
\href{mailto:schnell18@gmail.com}{\emph{Justin Zhang}} in \LaTeX\ for
better screen reading experience. The author of this document is
\emph{Vlad Krasnov}. You may read the document in its original form at
\href{https://blog.cloudflare.com/results-experimenting-brotli/}{https://blog.cloudflare.com/results-experimenting-brotli/}.

\newpage

Compression is one of the most important tools CloudFlare has to
accelerate website performance. Compressed content takes less time to
transfer, and consequently reduces load times. On expensive mobile data
plans, compression even saves money for consumers. However, compression
is not free—it comes at a price. It is one of the most compute expensive
operations our servers perform, and the better the compression rate we
want, the more effort we have to spend.

The most popular compression format on the web is gzip. We put a great
deal of effort into improving the performance of the gzip compression,
so we can perform compression on the fly with fewer CPU cycles. Recently
a potential replacement for gzip, called Brotli, was announced by
Google. Being early adopters for many technologies, we at CloudFlare
want to see for ourselves if it is as good as claimed.

This post takes a look at a bit of history behind gzip and Brotli,
followed by a performance comparison.

\newpage
\section{Compression 101}

Many popular lossless compression algorithms rely on LZ77 and Huffman
coding, so it’s important to have a basic understanding of these two
techniques before getting into gzip or Brotli.

\subsection{LZ77}

LZ77 is a simple technique developed by Abraham Lempel and Jacob Ziv in
1977 (hence the original name). Let's call the input to the algorithm a
string (a sequence of bytes, not necessarily letters) and each
consecutive sequence of bytes in the input a substring. LZ77 compresses
the input string by replacing some of its substrings by pointers (or
backreferences) to an identical substring previously encountered in the
input.

The pointer usually has the form of <length, distance>, where length
indicates the number of identical bytes found, and distance indicates
how many bytes separate the current occurrence of the substring from the
previous one. For example the string abcdeabcdf can be compressed with
LZ77 to abcde<4,5>f and aaaaaaaaaa can be compressed to simply a<9, 1>.
The decompressor when encountering a backreference will simply copy the
required number of bytes from the already decompressed output, which
makes it very fast for decompression. This is a nice illustration of
LZ77 from my previous blog
\href{https://blog.cloudflare.com/improving-compression-with-preset-deflate-dictionary/}{Improving
compression with a preset DEFLATE dictionary}:
\\\\

\begin{table}[H]
\ttfamily
\begin{tabular}{cccccccccccccccccccc}
L & i & t & t & l & e &  & b & u & n & n & y &  & F & o & o &  & F & o & o \\
W & e & n & t &  & h & o & p & p & i & n & g &  & t & h & r & o & u & g \\
h &  & t & h & e &  & f & o & r & e & s & t &  & S & c & o & o & p & i & n \\
g &  & u & p &  & t & h & e &  & f & i & e & l & d &  & m & i & c & e &  \\
A & n & d &  & b & o & p & p & i & n & g &  & t & h & e & m &  & o & n &  \\
t & h & e &  & h & e & a & d &  & D & o & w & n &  & c & a & m & e &  & t \\
h & e &  & G & o & o & d &  & F & a & i & r & y & , &  & a & n & d &  & s \\
h & e &  & s & a & i & d &  & " & L & i & t & t & l & e &  & b & u & n & n \\
y &  & F & o & o &  & F & o & o &  & I &  & d & o & n & ' & t &  & w & a \\
n & t &  & t & o &  & s & e & e &  & y & o & u &  & S & c & o & o & p & i \\
n & g &  & u & p &  & t & h & e &  & f & i & e & l & d &  & m & i & c & e \\
A & n & d &  & b & o & p & p & i & n & g &  & t & h & e & m &  & o & n \\
t & h & e &  & h & e & a & d & . & " &   &  &   &   &   &   &  &   &
\end{tabular}
\end{table}

Output (length tokens are blue, distance tokens are red): \\\\

\begin{table}[H]
\ttfamily
\begin{tabular}{cccccccccccccccccccc}
  L & i & t & t & l & e &  & b & u & n & n & y &  & F & o & o & \color{blue} 5 & \color{red} 4 & W & e \\
  n & t &  & h & o & p & p & i & n & g &  & t & h & r & o & u & g & h & \color{blue} 3 & \color{red} 8 \\
  e &  & f & o & r & e & s & t &  & S & c & o & o & \color{blue} 5 & \color{red} 28 & u & p & \color{blue} 6 & \color{red} 23 & i \\
  e & l & d &  & m & i & c & e &  & A & n & d &  & b & \color{blue} 9 & \color{red} 58 & e & m &  & o \\
  n & \color{blue} 5 & \color{red} 35 & h & e & a & d &  & D & o & w & n &  & c & a & m & e & \color{blue} 5 & \color{red} 19 & G \\
  o & o & d &  & F & a & i & r & y & , &  & a & \color{blue} 3 & \color{red} 55 & s & \color{blue} 3 & \color{red} 20 & s & a & i \\
  d &  & " & L & \color{blue} 20 & \color{red} 149 & I &  & d & o & n & ' & t &  & w & a & \color{blue} 3 & \color{red} 157 & t & o \\
  s & e & e &  & y & o & u & \color{blue} 56 & \color{red} 141 & . & " &  &  &  &  &  &  &  &  &
\end{tabular}
\end{table}

\subsection{Huffman Coding}

Huffman coding is another lossless compression algorithm. Developed by
David Huffman back in the 50s, it is used for many compression
algorithms, including JPEG. A Huffman code is a type of prefix coding,
where, given an alphabet and an input, frequently occurring characters
are replaced by shorter bit sequences and rarely occurring characters
are replaced with longer sequences.

The code can be expressed as a binary tree, where the leaf nodes are the
literals of the alphabet and the two edges from each node are marked
with 0 and 1. To decode the next character, the decompressor can parse
the tree from the root until it encounters a literal in the tree.

Each compression format uses Huffman coding differently, and for our
little example we will create a Huffman code for an alphabet that
includes only the literals and the length codes we actually used. To
start, we must count the frequency of each letter in the LZ77 compressed
text:

\begin{bashcode}
(space) - 19, o -14, e - 11, n - 8, t - 7, a - 6, d - 6, i - 6, 3 - 4, 5
- 4, h - 4, s - 4, u - 4, c - 3, m - 3, p - 3, r - 3, y - 3, (") - 2, F
- 2, L - 2, b - 2, g - 2, l - 2, w - 2, (') - 1, (,) - 1, (.) - 1, A -
1, D - 1, G - 1, I - 1, 9 - 1, 20 - 1, S - 1, 56 - 1, W - 1, 6 - 1, f -
1.
\end{bashcode}

We can then use the algorithm from Wikipedia to build this Huffman code (binary):

\begin{bashcode}
(space) - 101, o - 000, e - 1101, n - 0101, t - 0011, a - 11101, d -
11110, i - 11100, 3 - 01100, 5 - 01111, h - 10001, s - 11000, u - 10010,
c - 00100, m - 111111, p - 110011, r - 110010, y - 111110, (") - 011010,
F - 010000, L - 100000, b - 011101, g - 010011, l - 011100, w - 011011,
(') - 0100101, (,) - 1001110, (.) - 1001100, A - 0100011, D - 0100010, G
- 1000011, I - 0010110, 9 - 1000010, 20 - 0010111, S - 0010100, 56 -
0100100, W - 0010101, 6 - 1001101, f - 1001111.
\end{bashcode}

Here the most frequent letters - space and 'o' got the shortest codes,
only 3 bit long, whereas the letters that occur only once get 7 bit
codes. If we were to represent the alphabet of 256 bytes and some length
tokens, we would require 9 bits per every letter.

Now apply the code to the LZ77 output, (leaving the distance tokens
untouched) and we get:

\begin{bashcode}
  100000 11100 0011 0011 011100 1101 101 011101 10010 0101 0101 111110
  101 010000 000 000 01111 4 0010101 1101 0101 0011 101 10001 000 110011
  110011 11100 0101 010011 101 0011 10001 110010 000 10010 010011 01100
  8 10001 1101 101 1001111 000 110010 1101 11000 0011 101 0010100 00100
  000 000 01111 28 10010 110011 1001101 23 11100 1101 011100 11110 101
  111111 11100 00100 1101 101 0100011 0101 11110 101 011101 1000010 58
  1101 111111 101 000 0101 01111 35 10001 1101 11101 11110 101 0100010
  000 011011 0101 101 00100 11101 111111 1101 01111 19 1000011 000 000
  11110 101 010000 11101 11100 110010 111110 1001110 101 11101 01100 55
  11000 01100 20 11000 11101 11100 11110 101 011010 100000 0010111 149
  0010110 101 11110 000 0101 0100101 0011 101 011011 11101 01100 157
  0011 000 101 11000 1101 1101 101 111110 000 10010 0100100 141 1001100
  011010
\end{bashcode}

Assuming all the distance tokens require one byte to store, the total
output length is 898 bits. Compared to the 1356 bits we would require to
store the LZ77 output, and 2008 bits for the original input. We achieved
a compression ratio of 31\% which is very good. In reality this is not
the case, since we must also encode the Huffman tree we used, otherwise
it would be impossible to decompress the text.


\subsection{gzip}

The compression algorithm used in gzip is called "deflate," and it’s a
combination of the LZ77 and Huffman algorithms discussed above.

On the LZ77 side, gzip has a minimum size of 3 bytes and a maximum of
258 bytes for the length tokens and a maximum of 32768 bytes for the
distance tokens. The maximum distance also defines the sliding window
size the implementation uses for compression and decompression.

For Huffman coding, deflate has two alphabets. The first alphabet
includes the input literals ("letters" 0-255), the end-of-block symbol
(the "letter" 256) and the length tokens ("letters" 257-285). There are
only 29 letters to encode all the possible lengths and the tokens
265-284 will always have additional bits encoded in the stream to cover
the entire range from 3 to 258. For example the letter 257 indicates the
minimal length of 3, whereas the letter 265 will indicate the length 11
if followed by the bit 0, and the length 12 if followed by the bit 1.

The second alphabet is for the distance tokens only. Its letters are the
codewords 0 through 29. Similar to the length tokens, the codes 4-29 are
followed by 1 to 13 additional bits to cover the whole range from 1 to
32768.

Using two distinct alphabets makes a lot of sense, because it allows us
to represent the distance code with fewer bits while avoiding any
ambiguity, since the distance codes always follow the length codes.

The most common implementation of gzip compression is the zlib library. At
CloudFlare, we use a custom version of zlib
\href{https://blog.cloudflare.com/cloudflare-fights-cancer/}{optimized for our
servers}.

zlib has 9 preset quality settings for the deflate algorithm, labeled
from 1 to 9. It can also be run with quality set to 0, in which case it
does not perform any compression. These quality settings can be divided
into two categories:

\begin{itemize}
  \item Fast compression (levels 1-3): When a sufficiently long backreference
    is found, it is emitted immediately, and the search moves to the position
    at the end of the matched string. If the match is longer than a few bytes,
    the entire matched string will not be hashed, meaning its substrings will
    never be referenced in the future. Clearly, this reduces the compression
    ratio.
  \item Slow compression (levels 4-9): Here, every substring is hashed;
    therefore, any substring can be referenced in the future. In addition slow
    compression enables "lazy" matches. If at a given position a sufficiently
    long match is found, it will not be immediately emitted. Instead, the
    algorithm attempts to find a match at the next position. If a longer match
    is found, the algorithm will emit a single literal at the current position
    instead the shorter match, and continue to the next position. As a rule,
    this results in a better compression ratio than levels 1-3.

\end{itemize}

Other differences between the quality levels are how far to search for a
backreference and how long a match should be before stopping.

A very decent compression rate can be observed at levels 3-4, which are
also quite fast. Increasing compression quality from level 4 and up
gives incrementally smaller compression gains, while requiring
substantially more time. Often, levels 8 and 9 will produce similarly
compressed output, but level 9 will require more time to do it.

It is important to understand that the zlib implementation does not
guarantee the best compression possible with the format, even when the
quality setting is set to 9. Instead, it uses a set of heuristics and
optimizations that allow for very good compression at reasonable speed.

Better gzip compression can be achieved by the zopfli library, but it is
significantly slower than zlib.

\subsection{Brotli}

The Brotli format was developed by Google and has been refined for a
while now. Here at CloudFlare, we built an nginx module that performs
dynamic Brotli compression, and we deployed it on our test server that
supports HTTP2 and other new features.

To check the Brotli compression on the test server you can use the
nightly build of the Firefox browser, which also supports this format.

\subsection{Brotli, deflate, and gzip}

Brotli and deflate are very closely related. Brotli also uses the LZ77
and Huffman algorithms for compression. Both algorithms use a sliding
window for backreferences. Gzip uses a fixed size, 32KB window, and
Brotli can use any window size from 1KB to 16MB, in powers of 2 (minus
16 bytes). This means that the Brotli window can be up to 512 times
larger window than the deflate window. This difference is almost
irrelevant in web-server context, as text files larger than 32KB are the
minority.

Other differences include smaller minimal match length (2 bytes minimum
in Brotli, compared to 3 bytes minimum in deflate) and larger maximal
match length (16779333 bytes in Broli, compared to 258 bytes in
deflate).

\subsection{Static dictionary}

Brotli also features a static dictionary. The "dictionary" supported by
deflate can greatly improve compression, but has to be supplied
independently and can only be addressed as part of the sliding window.
The Brotli dictionary is part of the implementation and can be
referenced from anywhere in the stream, somewhat increasing its
efficiency for larger files. Moreover different transformations can be
applied to words of the dictionary effectively increasing its size.

\subsection{Context modeling}

Brotli also supports something called context modeling. Context modeling
is a feature that allows multiple Huffman trees for the same alphabet in
the same block. For example, in deflate each block consists of a series
of literals (bytes that could not be compressed by backreferencing) and
<length, distance> pairs that define a backreference for copying.
Literals and lengths form a single alphabet, while the distances are a
different alphabet.

In Brotli, each block is composed of "commands". A command consists of 3
parts. The first part of each command is a word <insert, copy>. "Insert"
defines the number of literals that will follow the word, and it may
have the value of 0. "Copy" defines the number of literals to copy from
a back reference. The word <insert,copy> is followed by a sequence of
"insert" literals: <lit> … <lit>. Finally the command ends with a
<distance>. Distance defines the backreference from which to copy the
previously defined number of bytes. Unlike the distance in deflate,
Brotli distance can have additional meanings, such as references to the
static dictionary, or references to recently used distances.

There are three alphabets here. One is for <lit>s and it simply covers
all the possible byte values from 0 to 255. The other one is for
<distance>s, and its size depends on the size of the sliding window and
other parameters. The third alphabet is for the <insert,copy> length
pairs, with 704 letters. Here <insert, copy> indicates a single letter
in the alphabet, as opposed to <length,distance> pairs in deflate where
length and distance are letters in distinct alphabets.

So why do we care about context modeling? It means that for any of the
alphabets—up to 256 different Huffman trees—can be used in the same
block. The switch between different trees is determined by "context".
This can be useful when the compressed file consists of different types
of characters. For example binary data interleaved with UTF-8 strings,
or a multilingual dictionary.

Whereas the basic idea behind Brotli remains identical to that of
deflate, the way the data is encoded is very different. Those
improvements allow for significantly better compression, but they also
require a significant amount of processing. To alleviate the performance
cost somewhat, Brotli drops the error detection CRC check present in
gzip.

\section{Benchmarking}

The heaviest operation in both deflate and Brotli is the search for
backward references. A higher level in zlib generally means that the
backward reference search will attempt to find a better match for
longer, but not necessarily succeed. This leads to significant increase
in processing time. In contrast, Brotli trades longer searches for
lighter operations that give better ROI, such as context modeling,
dictionary references and more efficient data encoding in general. For
that reason Brotli is, in theory, capable of outperforming zlib at some
stage for similar compression rates, as well as giving better
compression at its maximal levels.

We decided to put those theories to the test. At CloudFlare, one of the
primary use cases for Brotli/gzip is on-the-fly compression of textual
web assets like HTML, CSS, and JavaScript, so that’s what we’ll be
testing.

There is a tradeoff between compression speed and transfer speed. It is
only beneficial to increase your compression ratio if you can reduce the
number of bytes you have to transfer faster than you would actually
transfer them. Slower compression will actually slow the connection
down. CloudFlare currently uses our own zlib implementation with quality
set to 8, and that is our benchmarking baseline.

The benchmark set consists of 10,655 HTML, CSS, and JavaScript files.
The benchmarks were performed on an Intel 3.5GHz, E3-1241 v3 CPU.

The files were grouped into several size groups, since different
websites have different size characteristics, and we must optimize for
them all.

Although the quality setting in the Brotli implementation states the
possible values are 0 to 11, we didn't see any differences between 0 and
1, or between 10 and 11, therefore only quality settings 1 to 10 are
reported.

\subsection{Compression quality}

Compression quality is measured as (total size of all files after
compression)/(total size of all files before compression)*100\%. The
columns represent the size distribution of the HTML, CSS, and JavaScript
files used in the test in bytes (the file size ranges are shown using
the [x,y) notation).

\begin{table}[H]
\centering
\begin{threeparttable}
\begin{tabular}{ccccccccccc}
  \toprule
  \thead{} & \thead{[20,\\ 1024)} & \thead{[1024,\\ 2048)} & \thead{[2048,\\ 3072)}
           & \thead{[3072,\\ 4096)} & \thead{[4096,\\ 8192)} & \thead{[8192,\\ 16384)}
           & \thead{[16384,\\ 32768)} & \thead{[32768,\\ 65536)}
           & \thead{[65536,\\ +$\infty$)} & \thead{All \\ files} \\
  \midrule
  zlib 1 & 65.0\% & 46.8\% & 42.8\% & 38.7\% & 34.4\% & 32.0\% & 29.9\% & 31.1\% & 31.4\% & 31.5\% \\
  zlib 2 & 64.9\% & 46.5\% & 42.5\% & 38.3\% & 33.8\% & 31.4\% & 29.2\% & 30.3\% & 30.5\% & 30.6\% \\
  zlib 3 & 64.9\% & 46.4\% & 42.3\% & 38.0\% & 33.6\% & 31.1\% & 28.8\% & 29.8\% & 29.9\% & 30.1\% \\
  zlib 4 & 64.5\% & 45.8\% & 41.6\% & 37.1\% & 32.6\% & 30.0\% & 27.8\% & 28.5\% & 28.7\% & 28.9\% \\
  zlib 5 & 64.5\% & 45.5\% & 41.3\% & 36.7\% & 32.1\% & 29.4\% & 27.1\% & 27.8\% & 27.8\% & 28.0\% \\
  zlib 6 & 64.4\% & 45.5\% & 41.3\% & 36.7\% & 32.0\% & 29.3\% & 27.0\% & 27.6\% & 27.6\% & 27.8\% \\
  zlib 7 & 64.4\% & 45.5\% & 41.3\% & 36.6\% & 32.0\% & 29.2\% & 26.9\% & 27.5\% & 27.5\% & 27.7\% \\
  zlib 8 & 64.4\% & 45.5\% & 41.3\% & 36.6\% & 32.0\% & 29.2\% & 26.9\% & 27.5\% & 27.4\% & 27.7\% \\
  zlib 9 & 64.4\% & 45.5\% & 41.3\% & 36.6\% & 32.0\% & 29.2\% & 26.9\% & 27.5\% & 27.4\% & 27.7\% \\
  brotli 1 & 61.3\% & 46.6\% & 42.7\% & 38.4\% & 33.8\% & 30.8\% & 28.6\% & 29.5\% & 28.6\% & 29.0\% \\
  brotli 2 & 61.9\% & 46.9\% & 42.8\% & 38.3\% & 33.6\% & 30.6\% & 28.3\% & 29.1\% & 28.3\% & 28.6\% \\
  brotli 3 & 61.8\% & 46.8\% & 42.7\% & 38.2\% & 33.3\% & 30.4\% & 28.1\% & 28.9\% & 28.0\% & 28.4\% \\
  brotli 4 & 53.8\% & 40.8\% & 38.6\% & 34.8\% & 30.9\% & 28.7\% & 27.0\% & 28.0\% & 27.5\% & 27.7\% \\
  brotli 5 & 49.9\% & 37.7\% & 35.7\% & 32.3\% & 28.7\% & 26.6\% & 25.2\% & 26.2\% & 26.0\% & 26.1\% \\
  brotli 6 & 50.0\% & 37.7\% & 35.7\% & 32.3\% & 28.6\% & 26.5\% & 25.1\% & 26.0\% & 25.7\% & 25.9\% \\
  brotli 7 & 50.0\% & 37.6\% & 35.6\% & 32.3\% & 28.5\% & 26.4\% & 25.0\% & 25.9\% & 25.5\% & 25.7\% \\
  brotli 8 & 50.0\% & 37.6\% & 35.6\% & 32.3\% & 28.5\% & 26.4\% & 25.0\% & 25.9\% & 25.4\% & 25.6\% \\
  brotli 9 & 50.0\% & 37.6\% & 35.5\% & 32.2\% & 28.5\% & 26.4\% & 25.0\% & 25.8\% & 25.3\% & 25.5\% \\
  brotli 10 & 46.8\% & 33.4\% & 32.5\% & 29.4\% & 26.0\% & 23.9\% & 22.9\% & 23.8\% & 23.0\% & 23.3\% \\
  \bottomrule\addlinespace[1ex]
\end{tabular}
% \begin{tablenotes}
% \end{tablenotes}
\end{threeparttable}
\end{table}

Clearly, the compression possible by Brotli is significant. On average,
Brotli at the maximal quality setting produces 1.19X smaller results
than zlib at the maximal quality. For files smaller than 1KB the result
is 1.38X smaller on average, a very impressive improvement, that can
probably be attributed to the use of static dictionary.

\subsection{Compression speed}

Compression speed is measured as (total size of files before
compression)/(total time to compress all files) and is reported in MB/s.

\begin{table}[H]
\centering
\begin{threeparttable}
\begin{tabular}{ccccccccccc}
  \toprule
  \thead{} & \thead{[20,\\ 1024)} & \thead{[1024,\\ 2048)} & \thead{[2048,\\ 3072)}
           & \thead{[3072,\\ 4096)} & \thead{[4096,\\ 8192)} & \thead{[8192,\\ 16384)}
           & \thead{[16384,\\ 32768)} & \thead{[32768,\\ 65536)}
           & \thead{[65536,\\ +$\infty$)} & \thead{All \\ files} \\
  \midrule
  zlib 1 & 5.9 & 21.8 & 34.4 & 43.4 & 62.1 & 89.8 & 117.9 & 127.9 & 139.6 & 125.5 \\
  zlib 2 & 5.9 & 21.7 & 34.3 & 43.0 & 61.2 & 87.6 & 114.3 & 123.1 & 130.7 & 118.9 \\
  zlib 3 & 5.9 & 21.7 & 34.0 & 42.4 & 60.5 & 84.8 & 108.0 & 114.5 & 114.9 & 106.9 \\
  zlib 4 & 5.8 & 20.9 & 32.2 & 39.8 & 54.8 & 74.9 & 93.3 & 97.5 & 96.1 & 90.7 \\
  zlib 5 & 5.8 & 20.6 & 31.4 & 38.3 & 51.6 & 68.4 & 82.0 & 81.3 & 73.2 & 71.6 \\
  zlib 6 & 5.8 & 20.6 & 31.2 & 37.9 & 50.6 & 64.0 & 73.7 & 70.2 & 57.5 & 58.0 \\
  zlib 7 & 5.8 & 20.5 & 31.0 & 37.4 & 49.6 & 60.8 & 67.4 & 64.6 & 51.0 & 52.0 \\
  zlib 8 & 5.8 & 20.5 & 31.0 & 37.2 & 48.8 & 53.2 & 56.6 & 56.5 & 41.6 & 43.1 \\
  zlib 9 & 5.8 & 20.6 & 30.8 & 37.3 & 48.6 & 51.7 & 56.6 & 54.2 & 40.4 & 41.9 \\
  brotli 1 & 3.4 & 12.8 & 20.4 & 25.9 & 37.8 & 57.3 & 80.0 & 94.1 & 105.8 & 91.3 \\
  brotli 2 & 3.4 & 12.4 & 19.5 & 24.4 & 35.2 & 52.3 & 71.2 & 82.0 & 89.0 & 78.8 \\
  brotli 3 & 3.4 & 12.3 & 19.0 & 23.7 & 34.0 & 49.8 & 67.4 & 76.3 & 81.5 & 73.0 \\
  brotli 4 & 2.0 & 7.6 & 11.9 & 15.2 & 22.2 & 33.1 & 44.7 & 51.9 & 58.5 & 51.0 \\
  brotli 5 & 2.0 & 5.2 & 8.0 & 10.3 & 15.0 & 22.0 & 29.7 & 33.3 & 32.8 & 30.3 \\
  brotli 6 & 1.8 & 3.8 & 5.5 & 7.0 & 10.5 & 16.3 & 23.5 & 28.6 & 28.4 & 25.6 \\
  brotli 7 & 1.5 & 2.3 & 3.1 & 3.7 & 4.9 & 7.2 & 10.7 & 15.5 & 19.6 & 16.2 \\
  brotli 8 & 1.4 & 2.3 & 2.7 & 3.1 & 4.0 & 5.3 & 7.1 & 10.6 & 15.1 & 12.2 \\
  brotli 9 & 1.3 & 2.1 & 2.4 & 2.8 & 3.4 & 4.3 & 5.5 & 7.0 & 10.6 & 8.8 \\
  brotli 10 & 0.2 & 0.4 & 0.4 & 0.5 & 0.5 & 0.6 & 0.6 & 0.6 & 0.5 & 0.5 \\
  \bottomrule\addlinespace[1ex]
\end{tabular}
% \begin{tablenotes}
% \end{tablenotes}
\end{threeparttable}
\end{table}


On average for all files, we can see that Brotli at quality level 4 is
slightly faster than zlib at quality level 8 (and 9) while having
comparable compression ratio. However that is misleading. Most files are
smaller than 64KB, and if we look only at those files then Brotli 4 is
actually 1.48X slower than zlib level 8!

\subsection{Connection speedup}

For on-the-fly compression, the most important question is how much time
to invest in compression to make data transfer faster. Because
increasing compression quality only gives incremental improvement over a
given level, we need the added compressed bytes to outweigh the
additional time spent compressing those bytes.

Again, CloudFlare uses compression quality 8 with zlib, so that’s our
baseline. For each quality setting of Brotli starting at 4 (which is
somewhat comparable to zlib 8 in terms of both time and compression
ratio), we compute the added compression speed as: ((total size for zlib
8) - (total size after compression with Brotli))/((total time for
Brotli)-(total time for zlib 8)).

The results are reported in MB/s. Negative numbers indicate lower
compression ratio.

\begin{table}[H]
\centering
\begin{threeparttable}
\begin{tabular}{ccccccccccc}
  \toprule
  \thead{} & \thead{[20,\\ 1024)} & \thead{[1024,\\ 2048)} & \thead{[2048,\\ 3072)}
           & \thead{[3072,\\ 4096)} & \thead{[4096,\\ 8192)} & \thead{[8192,\\ 16384)}
           & \thead{[16384,\\ 32768)} & \thead{[32768,\\ 65536)}
           & \thead{[65536,\\ +$\infty$)} & \thead{All \\ files} \\
  \midrule
  brotli 4 & 0.33 & 0.56 & 0.52 & 0.47 & 0.42 & 0.44 & -0.24 & -2.86 & 0.02 & 0.00 \\
  brotli 5 & 0.44 & 0.55 & 0.60 & 0.61 & 0.71 & 0.97 & 1.01 & 1.08 & 2.24 & 1.58 \\
  brotli 6 & 0.36 & 0.36 & 0.37 & 0.37 & 0.45 & 0.63 & 0.69 & 0.86 & 1.52 & 1.12 \\
  brotli 7 & 0.28 & 0.20 & 0.19 & 0.18 & 0.19 & 0.23 & 0.24 & 0.34 & 0.72 & 0.52 \\
  brotli 8 & 0.26 & 0.20 & 0.17 & 0.15 & 0.15 & 0.17 & 0.15 & 0.21 & 0.48 & 0.35 \\
  brotli 9 & 0.25 & 0.19 & 0.15 & 0.13 & 0.13 & 0.13 & 0.11 & 0.13 & 0.30 & 0.24 \\
  brotli 10 & 0.03 & 0.04 & 0.04 & 0.04 & 0.03 & 0.03 & 0.02 & 0.02 & 0.02 & 0.02 \\
  \bottomrule\addlinespace[1ex]
\end{tabular}
% \begin{tablenotes}
% \end{tablenotes}
\end{threeparttable}
\end{table}


Those numbers are quite low, due to the slow speed of Brotli. To get any
speedup we really want to see those numbers being greater than the
connection speed. It seems that for files greater than 64KB, Brotli at
quality setting 5 can speed up slow connections.

Keep in mind, that on a real server, compression is only one of many
tasks that share the CPU, and compression speeds would be slower there.

\section{Conclusions}

The current state of Brotli gives us some mixed impressions. There is no
yes/no answer to the question "Is Brotli better than gzip?". It
definitely looks like a big win for static content compression, but on
the web where the content is dynamic we also need to consider on-the-fly
compression.

The way I see it, Brotli already has an advantage over zlib for large
files (larger than 64KB) on slow connections. However, those constitute
only 20\% of our sampled dataset (and 80\% of the total size).

Our Brotli module has a minimal size setting for Brotli compression that
allows us to use gzip for smaller files and Brotli only for large ones.

It is important to remember that zlib has the advantage of being the
optimization target for years by the entire web community, while Brotli
is the development effort of a small but capable and talented team.
There is no doubt that the current implementation will only improve with
time.

\end{document}
% vim: set ai nu nobk expandtab ts=4 sw=2 tw=72 syntax=tex :
