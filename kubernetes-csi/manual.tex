%!TEX TS-program = xelatex
%!TEX encoding = UTF-8 Unicode

\documentclass[12pt]{article}
\usepackage{geometry}
\usepackage{float}
\usepackage{amssymb}
\usepackage{amsmath}
\usepackage{tikz}
\usepackage{minted}
\usepackage[font={small,it}]{caption}
\usepackage{booktabs,makecell,multirow,threeparttable}
\usepackage{graphicx}
\graphicspath{{./images/}}

\usepackage{pdflscape}
\usepackage{afterpage}
\usepackage{capt-of}

\usepackage{fancyhdr}
\setlength{\headheight}{25pt}
\pagestyle{fancy}
\fancyhf{}
\fancyhead[L]{\leftmark}
\fancyhead[R]{\rightmark}
\fancyfoot[C]{\thepage}

\usepackage[utf8]{inputenc}
\usepackage{fontspec,xltxtra,xunicode}
\usepackage[T1]{fontenc}
\usepackage{biolinum}
\usepackage{libertineRoman} % or any other font package
\usepackage[
  colorlinks=true,
  linkcolor=blue,
  bookmarksnumbered=true,
  bookmarksopen=true]{hyperref}

\newcommand{\mytitle}[1]{
  {\Huge #1}\par
}

\newcommand{\mysubtitle}[1]{
  {\emph{#1}}\par
}

\definecolor{titlepagecolor}{cmyk}{1,.60,0,.40}

\DeclareFixedFont{\titlefont}{T1}{ppl}{b}{it}{1.0in}

\newminted{bash}{
  autogobble,
  breaklines,
  frame=leftline,
  framerule=1.2pt,
  framesep=1em,
  linenos,
  fontsize=\footnotesize
}

%!TEX TS-program = lualatex
%!TEX r = UTF-8 Unicode
%!TeX root = manual.tex

\newcommand\titlepagedecoration{%
\begin{tikzpicture}[remember picture,overlay,shorten >= -10pt]

\coordinate (aux1) at ([yshift=-15pt]current page.north east);
\coordinate (aux2) at ([yshift=-410pt]current page.north east);
\coordinate (aux3) at ([xshift=-4.5cm]current page.north east);
\coordinate (aux4) at ([yshift=-150pt]current page.north east);

\begin{scope}[titlepagecolor!40,line width=12pt,rounded corners=12pt]
\draw
  (aux1) -- coordinate (a)
  ++(225:5) --
  ++(-45:5.1) coordinate (b);
\draw[shorten <= -10pt]
  (aux3) --
  (a) --
  (aux1);
\draw[opacity=0.6,titlepagecolor,shorten <= -10pt]
  (b) --
  ++(225:2.2) --
  ++(-45:2.2);
\end{scope}
\draw[titlepagecolor,line width=8pt,rounded corners=8pt,shorten <= -10pt]
  (aux4) --
  ++(225:0.8) --
  ++(-45:0.8);
\begin{scope}[titlepagecolor!70,line width=6pt,rounded corners=8pt]
\draw[shorten <= -10pt]
  (aux2) --
  ++(225:3) coordinate[pos=0.45] (c) --
  ++(-45:3.1);
\draw
  (aux2) --
  (c) --
  ++(135:2.5) --
  ++(45:2.5) --
  ++(-45:2.5) coordinate[pos=0.3] (d);
\draw
  (d) -- +(45:1);
\end{scope}
\end{tikzpicture}%
}


\begin{document}

\begin{titlepage}
\begin{minipage}{0.8\linewidth}
  \vspace*{4cm}
  \begin{center}
    \noindent
    \mytitle{Container Storage Interface (CSI)}
    \rule{\linewidth}{0.2ex}\par
    \mysubtitle{Jie Yu \\ Saad Ali \\ Jame DeFelice \\ typeset by Justin Zhang}
  \end{center}
\end{minipage}
\null \vfill
\hfill
\begin{minipage}{0.55\linewidth}
  \begin{flushright}
    % \printauthor
  \end{flushright}
\end{minipage}
%
% \begin{minipage}{0.02\linewidth}
%   \rule{1pt}{260pt}
% \end{minipage}
\titlepagedecoration
\end{titlepage}

\newpage
\tableofcontents
% \newpage
% \listoffigures
% \newpage
% \listoftables
% \newpage
% \listoflistings

\newpage
This document is typesetted by
\href{mailto:schnell18@gmail.com}{\emph{Justin Zhang}} in \LaTeX\ for
better screen reading experience. The author of this document is
\emph{Jie Yu}, \emph{Saad Ali} and \emph{James DeFelice}. You may read
the document in its original form at \\
\href{https://github.com/container-storage-interface/spec/blob/master/spec.md}{https://github.com/container-storage-interface/spec/blob/master/spec.md}.

\newpage
\section{Notational Conventions}
The keywords "MUST", "MUST NOT", "REQUIRED", "SHALL", "SHALL NOT",
"SHOULD", "SHOULD NOT", "RECOMMENDED", "NOT RECOMMENDED", "MAY", and
"OPTIONAL" are to be interpreted as described in RFC 2119 (Bradner, S.,
"Key words for use in RFCs to Indicate Requirement Levels", BCP 14,
\href{http://tools.ietf.org/html/rfc2119}{RFC 2119}, March 1997).

The key words "unspecified", "undefined", and "implementation-defined"
are to be interpreted as described in the
\href{http://www.open-std.org/jtc1/sc22/wg14/www/C99RationaleV5.10.pdf\#page=18}{rationale
for the C99 standard}.

An implementation is not compliant if it fails to satisfy one or more of
the MUST, REQUIRED, or SHALL requirements for the protocols it
implements. An implementation is compliant if it satisfies all the MUST,
REQUIRED, and SHALL requirements for the protocols it implements.

\newpage
\section{Terminology}

  \begin{tabular}{lp{24em}}
    \bfseries Volume	& A unit of storage that will be made available \\
    \bfseries Block  Volume &	A volume that will appear as a block
    inside of a CO-managed container, via the CSI. device inside the container. \\
    \bfseries Mounted Volume	& A volume that will be mounted using the
    specified file system and appear as a directory inside the
    container. \\
    \bfseries CO	& Container Orchestration system, communicates with
    Plugins using CSI service RPCs. \\
    \bfseries SP	& Storage Provider, the vendor of a CSI plugin
    implementation. \\
    \bfseries RPC	& \href{https://en.wikipedia.org/wiki/Remote\_procedure\_call}{Remote Procedure Call}. \\
    \bfseries Node	& A host where the user workload will be running,
    uniquely identifiable from the perspective of a Plugin by a node ID.
    \\
    \bfseries Plugin	& Aka \emph{plugin implementation}, a gRPC endpoint
    that implements the CSI Services. \\
    \bfseries Plugin  Supervisor&	Process that governs the lifecycle
    of a Plugin, MAY be the CO. \\
    \bfseries Workload	& The atomic unit of "work" scheduled by a CO.
    This MAY be a container or a collection of containers. \\

\end{tabular}

\newpage
\section{Objective}

To define an industry standard “Container Storage Interface” (CSI) that
will enable storage vendors (SP) to develop a plugin once and have it
work across a number of container orchestration (CO) systems.

\subsection{Goals in MVP}

The Container Storage Interface (CSI) will

\begin{itemize}
  \item Enable SP authors to write one CSI compliant Plugin that
    \emph{just works} across all COs that implement CSI.
  \item Define API (RPCs) that enable:
  \begin{itemize}
    \item Dynamic provisioning and deprovisioning of a volume.
    \item Attaching or detaching a volume from a node.
    \item Mounting/unmounting a volume from a node.
    \item Consumption of both block and mountable volumes.
    \item Local storage providers (e.g., device mapper, lvm).
    \item Creating and deleting a snapshot (source of the snapshot is a volume).
    \item Provisioning a new volume from a snapshot (reverting snapshot,
      where data in the original volume is erased and replaced with data
      in the snapshot, is out of scope).
  \end{itemize}
  \item Define plugin protocol RECOMMENDATIONS.
  \begin{itemize}
    \item Describe a process by which a Supervisor configures a Plugin.
    \item Container deployment considerations (CAP\_SYS\_ADMIN, mount namespace, etc.).
  \end{itemize}
\end{itemize}

\subsection{Non-Goals in MVP}

The Container Storage Interface (CSI) explicitly will not define,
provide, or dictate:

\begin{itemize}
  \item Specific mechanisms by which a Plugin Supervisor manages the
    lifecycle of a Plugin, including:
  \begin{itemize}
    \item How to maintain state (e.g. what is attached, mounted, etc.).
    \item How to deploy, install, upgrade, uninstall, monitor, or
      respawn (in case of unexpected termination) Plugins.
  \end{itemize}
  \item A first class message structure/field to represent "grades of
    storage" (aka "storage class").
  \item Protocol-level authentication and authorization.
  \item Packaging of a Plugin.
  \item POSIX compliance: CSI provides no guarantee that volumes
    provided are POSIX compliant filesystems. Compliance is determined
    by the Plugin implementation (and any backend storage system(s) upon
    which it depends). CSI SHALL NOT obstruct a Plugin Supervisor or CO
    from interacting with Plugin-managed volumes in a POSIX-compliant
    manner.
\end{itemize}

\newpage
\section{Solution Overview}

This specification defines an interface along with the minimum
operational and packaging recommendations for a storage provider (SP) to
implement a CSI compatible plugin. The interface declares the RPCs that
a plugin MUST expose: this is the \emph{primary focus} of the CSI
specification. Any operational and packaging recommendations offer
additional guidance to promote cross-CO compatibility.

\subsection{Architecture}

The primary focus of this specification is on the \emph{protocol}
between a CO and a Plugin. It SHOULD be possible to ship cross-CO
compatible Plugins for a variety of deployment architectures. A CO
SHOULD be equipped to handle both centralized and headless plugins, as
well as split-component and unified plugins. Several of these
possibilities are illustrated in the following figures.

%%TODO: use tikz to draw diagrams
\begin{bashcode}
                             CO "Master" Host
+-------------------------------------------+
|                                           |
|  +------------+           +------------+  |
|  |     CO     |   gRPC    | Controller |  |
|  |            +----------->   Plugin   |  |
|  +------------+           +------------+  |
|                                           |
+-------------------------------------------+

                            CO "Node" Host(s)
+-------------------------------------------+
|                                           |
|  +------------+           +------------+  |
|  |     CO     |   gRPC    |    Node    |  |
|  |            +----------->   Plugin   |  |
|  +------------+           +------------+  |
|                                           |
+-------------------------------------------+

Figure 1: The Plugin runs on all nodes in the cluster: a centralized
Controller Plugin is available on the CO master host and the Node
Plugin is available on all of the CO Nodes.
\end{bashcode}

\begin{bashcode}
                            CO "Node" Host(s)
+-------------------------------------------+
|                                           |
|  +------------+           +------------+  |
|  |     CO     |   gRPC    | Controller |  |
|  |            +--+-------->   Plugin   |  |
|  +------------+  |        +------------+  |
|                  |                        |
|                  |                        |
|                  |        +------------+  |
|                  |        |    Node    |  |
|                  +-------->   Plugin   |  |
|                           +------------+  |
|                                           |
+-------------------------------------------+

Figure 2: Headless Plugin deployment, only the CO Node hosts run
Plugins. Separate, split-component Plugins supply the Controller
Service and the Node Service respectively.
\end{bashcode}


\begin{bashcode}
                            CO "Node" Host(s)
+-------------------------------------------+
|                                           |
|  +------------+           +------------+  |
|  |     CO     |   gRPC    | Controller |  |
|  |            +----------->    Node    |  |
|  +------------+           |   Plugin   |  |
|                           +------------+  |
|                                           |
+-------------------------------------------+

Figure 3: Headless Plugin deployment, only the CO Node hosts run
Plugins. A unified Plugin component supplies both the Controller
Service and Node Service.
\end{bashcode}

\begin{bashcode}
                            CO "Node" Host(s)
+-------------------------------------------+
|                                           |
|  +------------+           +------------+  |
|  |     CO     |   gRPC    |    Node    |  |
|  |            +----------->   Plugin   |  |
|  +------------+           +------------+  |
|                                           |
+-------------------------------------------+

Figure 4: Headless Plugin deployment, only the CO Node hosts run
Plugins. A Node-only Plugin component supplies only the Node Service.
Its GetPluginCapabilities RPC does not report the CONTROLLER_SERVICE
capability.
\end{bashcode}

\subsection{Volume Lifecycle}

\begin{bashcode}
   CreateVolume +------------+ DeleteVolume
 +------------->|  CREATED   +--------------+
 |              +---+----^---+              |
 |       Controller |    | Controller       v
+++         Publish |    | Unpublish       +++
|X|          Volume |    | Volume          | |
+-+             +---v----+---+             +-+
                | NODE_READY |
                +---+----^---+
               Node |    | Node
            Publish |    | Unpublish
             Volume |    | Volume
                +---v----+---+
                | PUBLISHED  |
                +------------+

Figure 5: The lifecycle of a dynamically provisioned volume, from
creation to destruction.
\end{bashcode}

\begin{bashcode}
   CreateVolume +------------+ DeleteVolume
 +------------->|  CREATED   +--------------+
 |              +---+----^---+              |
 |       Controller |    | Controller       v
+++         Publish |    | Unpublish       +++
|X|          Volume |    | Volume          | |
+-+             +---v----+---+             +-+
                | NODE_READY |
                +---+----^---+
               Node |    | Node
              Stage |    | Unstage
             Volume |    | Volume
                +---v----+---+
                |  VOL_READY |
                +---+----^---+
               Node |    | Node
            Publish |    | Unpublish
             Volume |    | Volume
                +---v----+---+
                | PUBLISHED  |
                +------------+

Figure 6: The lifecycle of a dynamically provisioned volume, from
creation to destruction, when the Node Plugin advertises the
STAGE_UNSTAGE_VOLUME capability.
\end{bashcode}

\begin{bashcode}
    Controller                  Controller
       Publish                  Unpublish
        Volume  +------------+  Volume
 +------------->+ NODE_READY +--------------+
 |              +---+----^---+              |
 |             Node |    | Node             v
+++         Publish |    | Unpublish       +++
|X| <-+      Volume |    | Volume          | |
+++   |         +---v----+---+             +-+
 |    |         | PUBLISHED  |
 |    |         +------------+
 +----+
   Validate
   Volume
   Capabilities

Figure 7: The lifecycle of a pre-provisioned volume that requires
controller to publish to a node (`ControllerPublishVolume`) prior to
publishing on the node (`NodePublishVolume`).
\end{bashcode}

\begin{bashcode}
       +-+  +-+
       |X|  | |
       +++  +^+
        |    |
   Node |    | Node
Publish |    | Unpublish
 Volume |    | Volume
    +---v----+---+
    | PUBLISHED  |
    +------------+

Figure 8: Plugins MAY forego other lifecycle steps by contraindicating
them via the capabilities API. Interactions with the volumes of such
plugins is reduced to `NodePublishVolume` and `NodeUnpublishVolume`
calls.
\end{bashcode}

The above diagrams illustrate a general expectation with respect to how
a CO MAY manage the lifecycle of a volume via the API presented in this
specification. Plugins SHOULD expose all RPCs for an interface:
Controller plugins SHOULD implement all RPCs for the Controller service.
Unsupported RPCs SHOULD return an appropriate error code that indicates
such (e.g. CALL\_NOT\_IMPLEMENTED). The full list of plugin capabilities
is documented in the ControllerGetCapabilities and NodeGetCapabilities
RPCs.

\newpage
\section{Container Storage Interface}
This section describes the interface between COs and Plugins.
\subsection{RPC Interface}

A CO interacts with an Plugin through RPCs. Each SP MUST provide: \\

\begin{tabular}{lp{24em}}
    \bfseries Node Plugin & A gRPC endpoint serving CSI RPCs that MUST be run
    on the Node whereupon an SP-provisioned volume will be published. \\
    \bfseries Controller Plugin & A gRPC endpoint serving CSI RPCs that
    MAY be run anywhere. \\
\end{tabular}

\inputminted[breaklines,frame=lines,linenos,fontsize=\footnotesize]{proto}{src/opts.proto}

In some circumstances a single gRPC endpoint MAY serve all CSI RPCs (see
Figure 3 in Architecture).

There are three sets of RPCs: \\

\begin{tabular}{lp{24em}}
    \bfseries Identity Service & Both the Node Plugin and the Controller Plugin
    MUST implement this sets of RPCs. \\
    \bfseries Controller Service & The Controller Plugin MUST implement this sets of RPCs. \\
    \bfseries Node Service & The Node Plugin MUST implement this sets of RPCs. \\
\end{tabular}

\inputminted[breaklines,frame=lines,linenos,fontsize=\footnotesize]{proto}{src/rpcs.proto}

\subsubsection{Concurrency}

In general the Cluster Orchestrator (CO) is responsible for ensuring
that there is no more than one call “in-flight” per volume at a given
time. However, in some circumstances, the CO MAY lose state (for example
when the CO crashes and restarts), and MAY issue multiple calls
simultaneously for the same volume. The plugin SHOULD handle this as
gracefully as possible. The error code ABORTED MAY be returned by the
plugin in this case (see the \nameref{sec:error-scheme} section for details).

\subsubsection{Field Requirements}

The requirements documented herein apply equally and without exception,
unless otherwise noted, for the fields of all protobuf message types
defined by this specification. Violation of these requirements MAY
result in RPC message data that is not compatible with all CO, Plugin,
and/or CSI middleware implementations.

\subsubsection{Size Limits}

CSI defines general size limits for fields of various types (see table
below). The general size limit for a particular field MAY be overridden
by specifying a different size limit in said field's description. Unless
otherwise specified, fields SHALL NOT exceed the limits documented here.
These limits apply for messages generated by both COs and plugins.

\begin{table}[H]
\centering
\begin{threeparttable}
  \begin{tabular}{ll}
  \toprule
  \thead{\bfseries Size} & \thead{\bfseries Field Type} \\
  \midrule
  128 bytes & \verb=string= \\
  4 KiB & \verb=map<string, string>= \\
  \bottomrule\addlinespace[1ex]
\end{tabular}
% \begin{tablenotes}
% \end{tablenotes}
\end{threeparttable}
  \caption{Size Limites}
  \label{tab:field-size-limit}
\end{table}


\subsubsection{REQUIRED vs. OPTIONAL}
\begin{itemize}
  \item A field noted as REQUIRED MUST be specified, subject to any
    per-RPC caveats; caveats SHOULD be rare.
  \item A repeated or map field listed as REQUIRED MUST contain at least 1 element.
  \item A field noted as OPTIONAL MAY be specified and the specification
    SHALL clearly define expected behavior for the default, zero-value
    of such fields.
\end{itemize}

Scalar fields, even REQUIRED ones, will be defaulted if not specified
and any field set to the default value will not be serialized over the
wire as per
\href{https://developers.google.com/protocol-buffers/docs/proto3\#default}{proto3}.

\subsubsection{Timeouts}

Any of the RPCs defined in this spec MAY timeout and MAY be retried. The
CO MAY choose the maximum time it is willing to wait for a call, how
long it waits between retries, and how many time it retries (these
values are not negotiated between plugin and CO).

Idempotency requirements ensure that a retried call with the same fields
continues where it left off when retried. The only way to cancel a call
is to issue a "negation" call if one exists. For example, issue a
ControllerUnpublishVolume call to cancel a pending
ControllerPublishVolume operation, etc. In some cases, a CO MAY NOT be
able to cancel a pending operation because it depends on the result of
the pending operation in order to execute the "negation" call. For
example, if a CreateVolume call never completes then a CO MAY NOT have
the volume\_id to call DeleteVolume with.

\subsection{Error Scheme} \label{sec:error-scheme}

All CSI API calls defined in this spec MUST return a
\href{https://github.com/grpc/grpc/blob/master/src/proto/grpc/status/status.proto}{standard
gRPC status}. Most gRPC libraries provide helper methods to set and read
the status fields.

The status code MUST contain a
\href{https://github.com/grpc/grpc-go/blob/master/codes/codes.go}{canonical
error code}. COs MUST handle all valid error codes. Each RPC defines a
set of gRPC error codes that MUST be returned by the plugin when
specified conditions are encountered. In addition to those, if the
conditions defined below are encountered, the plugin MUST return the
associated gRPC error code.

{%
  \begin{center}
  \pagestyle{empty}
  \begin{landscape}
  \begin{table}[H]
  \begin{threeparttable}
    \begin{tabular}{p{7em}p{12em}p{20em}p{20em}}
    \toprule
    \thead{\bfseries Condition} & \thead{\bfseries gRPC Code} & 
    \thead{\bfseries Description} & \thead{\bfseries Recovery Behavior} \\
    \midrule
  Missing required field & 3 INVALID\_ARGUMENT & Indicates that a required
      field is missing from the request. More human-readable information
      MAY be provided in the status.message field. & Caller MUST fix the
      request by adding the missing required field before retrying. \\
  Invalid or unsupported field in the request & 3 INVALID\_ARGUMENT &
      Indicates that the one or more fields in this field is either not
      allowed by the Plugin or has an invalid value. More human-readable
      information MAY be provided in the gRPC status.message field. &
      Caller MUST fix the field before retrying. \\
  Permission denied & 7 PERMISSION\_DENIED & The Plugin is able to derive
      or otherwise infer an identity from the secrets present within an
      RPC, but that identity does not have permission to invoke the RPC. &
      System administrator SHOULD ensure that requisite permissions are
      granted, after which point the caller MAY retry the attempted RPC.
      \\
  Operation pending for volume & 10 ABORTED & Indicates that there is
      already an operation pending for the specified volume. In general
      the Cluster Orchestrator (CO) is responsible for ensuring that there
      is no more than one call "in-flight" per volume at a given time.
      However, in some circumstances, the CO MAY lose state (for example
      when the CO crashes and restarts), and MAY issue multiple calls
      simultaneously for the same volume. The Plugin, SHOULD handle this
      as gracefully as possible, and MAY return this error code to reject
      secondary calls. & Caller SHOULD ensure that there are no other
      calls pending for the specified volume, and then retry with
      exponential back off. \\
  Not authenticated & 16 UNAUTHENTICATED & The invoked RPC does not carry
      secrets that are valid for authentication. & Caller SHALL either fix
      the secrets provided in the RPC, or otherwise regalvanize said
      secrets such that they will pass authentication by the Plugin for
      the attempted RPC, after which point the caller MAY retry the
      attempted RPC \\
  Call not implemented & 12 UNIMPLEMENTED & The invoked RPC is not
      implemented by the Plugin or disabled in the Plugin's current mode
      of operation. & Caller MUST NOT retry. Caller MAY call
      GetPluginCapabilities, ControllerGetCapabilities, or
      NodeGetCapabilities to discover Plugin capabilities. \\
    \bottomrule\addlinespace[1ex]
  \end{tabular}
  % \begin{tablenotes}
  % \end{tablenotes}
  \end{threeparttable}
    \caption{gRPC Code Explained}
    \label{tab:grpc-code-explained}
  \end{table}
  \end{landscape}
  \end{center}
}


\subsection{Secrets Requirements}

Secrets MAY be required by plugin to complete a RPC request. A secret is
a string to string map where the key identifies the name of the secret
(e.g. "username" or "password"), and the value contains the secret data
(e.g. "bob" or "abc123"). Each key MUST consist of alphanumeric
characters, '-', '\_' or '.'. Each value MUST contain a valid string. An
SP MAY choose to accept binary (non-string) data by using a
binary-to-text encoding scheme, like base64. An SP SHALL advertise the
requirements for required secret keys and values in documentation. CO
SHALL permit passing through the required secrets. A CO MAY pass the
same secrets to all RPCs, therefore the keys for all unique secrets that
an SP expects MUST be unique across all CSI operations. This information
is sensitive and MUST be treated as such (not logged, etc.) by the CO.

\subsection{Identity Service RPC}
Identity service RPCs allow a CO to query a plugin for capabilities,
health, and other metadata. The general flow of the success case MAY be
as follows (protos illustrated in YAML for brevity):

\begin{enumerate}
  \item CO queries metadata via Identity RPC.

    \begin{bashcode}
       # CO --(GetPluginInfo)--> Plugin
       request:
       response:
          name: org.foo.whizbang.super-plugin
          vendor_version: blue-green
          manifest:
            baz: qaz
    \end{bashcode}


  \item CO queries available capabilities of the plugin.

    \begin{bashcode}
       # CO --(GetPluginCapabilities)--> Plugin
       request:
       response:
         capabilities:
           - service:
               type: CONTROLLER_SERVICE
    \end{bashcode}


  \item CO queries the readiness of the plugin.

    \begin{bashcode}
       # CO --(Probe)--> Plugin
       request:
       response: {}
    \end{bashcode}


\end{enumerate}

\subsubsection{GetPluginInfo}

\inputminted[breaklines,frame=lines,linenos,fontsize=\footnotesize]{proto}{src/getplugininfo.proto}

{\bfseries{GetPluginInfo Errors}} \\

If the plugin is unable to complete the GetPluginInfo call successfully,
it MUST return a non-ok gRPC code in the gRPC status.

\subsubsection{GetPluginCapabilities}

This REQUIRED RPC allows the CO to query the supported capabilities of
the Plugin "as a whole": it is the grand sum of all capabilities of all
instances of the Plugin software, as it is intended to be deployed. All
instances of the same version (see \verb=vendor_version= of
\verb=GetPluginInfoResponse=) of the Plugin SHALL return the same set of
capabilities, regardless of both: (a) where instances are deployed on
the cluster as well as; (b) which RPCs an instance is serving.

\inputminted[breaklines,frame=lines,linenos,fontsize=\footnotesize]{proto}{src/getplugincapabilites.proto}

{\bfseries{GetPluginCapabilities Errors}} \\

If the plugin is unable to complete the GetPluginCapabilities call
successfully, it MUST return a non-ok gRPC code in the gRPC status.


\subsubsection{Probe}
A Plugin MUST implement this RPC call. The primary utility of the Probe
RPC is to verify that the plugin is in a healthy and ready state. If an
unhealthy state is reported, via a non-success response, a CO MAY take
action with the intent to bring the plugin to a healthy state. Such
actions MAY include, but SHALL NOT be limited to, the following:

\begin{itemize}
  \item Restarting the plugin container, or
  \item Notifying the plugin supervisor.
\end{itemize}

The Plugin MAY verify that it has the right configurations, devices,
dependencies and drivers in order to run and return a success if the
validation succeeds. The CO MAY invoke this RPC at any time. A CO MAY
invoke this call multiple times with the understanding that a plugin's
implementation MAY NOT be trivial and there MAY be overhead incurred by
such repeated calls. The SP SHALL document guidance and known
limitations regarding a particular Plugin's implementation of this RPC.
For example, the SP MAY document the maximum frequency at which its
Probe implementation SHOULD be called.

\inputminted[breaklines,frame=lines,linenos,fontsize=\footnotesize]{proto}{src/probe.proto}

{\bfseries{Probe Errors}} \\

If the plugin is unable to complete the Probe call successfully, it MUST
return a non-ok gRPC code in the gRPC status. If the conditions defined
below are encountered, the plugin MUST return the specified gRPC error
code. The CO MUST implement the specified error recovery behavior when
it encounters the gRPC error code.

  \begin{table}[H]
  \begin{threeparttable}
    \begin{tabular}{p{5em}p{11em}p{8em}p{8em}}
    \toprule
    \thead{\bfseries Condition} & \thead{\bfseries gRPC Code} & 
    \thead{\bfseries Description} & \thead{\bfseries Recovery Behavior} \\
    \midrule
      Plugin not healthy &	9 FAILED\_PRECONDITION &	Indicates that the
      plugin is not in a healthy/ready state. & Caller SHOULD assume the
      plugin is not healthy and that future RPCs MAY fail because of
      this condition. \\
      Missing required dependency &	9 FAILED\_PRECONDITION & Indicates
      that the plugin is missing one or more required dependency. &
      Caller MUST assume the plugin is not healthy. \\
    \bottomrule\addlinespace[1ex]
  \end{tabular}
  % \begin{tablenotes}
  % \end{tablenotes}
  \end{threeparttable}
    \caption{Probe gRPC Code Explained}
    \label{tab:probe-grpc-code-explained}
  \end{table}

\subsection{Controller Service RPC}
\subsubsection{CreateVolume}
A Controller Plugin MUST implement this RPC call if it has
\verb=CREATE_DELETE_VOLUME= controller capability. This RPC will be called by
the CO to provision a new volume on behalf of a user (to be consumed as
either a block device or a mounted filesystem).

This operation MUST be idempotent. If a volume corresponding to the
specified volume name already exists, is accessible from
\verb=accessibility_requirements=, and is compatible with the specified
\verb=capacity_range=, \verb=volume_capabilities= and parameters in the
CreateVolumeRequest, the Plugin MUST reply 0 OK with the corresponding
CreateVolumeResponse.

Plugins MAY create 3 types of volumes:

\begin{itemize}
  \item Empty volumes. When plugin supports \verb=CREATE_DELETE_VOLUME= OPTIONAL capability.
  \item From an existing snapshot. When plugin supports
    \verb=CREATE_DELETE_VOLUME= and \verb=CREATE_DELETE_SNAPSHOT=
    OPTIONAL capabilities.
  \item From an existing volume. When plugin supports cloning, and
    reports the OPTIONAL capabilities \verb=CREATE_DELETE_VOLUME= and
    \verb=CLONE_VOLUME=.
\end{itemize}

If CO requests a volume to be created from existing snapshot or volume
and the requested size of the volume is larger than the original
snapshotted (or cloned volume), the Plugin can either refuse such a call
with \verb=OUT_OF_RANGE= error or MUST provide a volume that, when presented to
a workload by NodePublish call, has both the requested (larger) size and
contains data from the snapshot (or original volume). Explicitly, it's
the responsibility of the Plugin to resize the filesystem of the newly
created volume at (or before) the NodePublish call, if the volume has
VolumeCapability access type MountVolume and the filesystem resize is
required in order to provision the requested capacity.

\inputminted[breaklines,frame=lines,linenos,fontsize=\footnotesize]{proto}{src/createvolume.proto}

{\bfseries{CreateVolume Errors}} \\
If the plugin is unable to complete the CreateVolume call successfully,
it MUST return a non-ok gRPC code in the gRPC status. If the conditions
defined below are encountered, the plugin MUST return the specified gRPC
error code. The CO MUST implement the specified error recovery behavior
when it encounters the gRPC error code.

  \begin{table}[H]
  \begin{threeparttable}
    \begin{tabular}{p{5em}p{11em}p{8em}p{8em}}
    \toprule
    \thead{\bfseries Condition} & \thead{\bfseries gRPC Code} & 
    \thead{\bfseries Description} & \thead{\bfseries Recovery Behavior} \\
    \midrule
      Plugin not healthy &	9 FAILED\_PRECONDITION &	Indicates that the
      plugin is not in a healthy/ready state. & Caller SHOULD assume the
      plugin is not healthy and that future RPCs MAY fail because of
      this condition. \\
      Missing required dependency &	9 FAILED\_PRECONDITION & Indicates
      that the plugin is missing one or more required dependency. &
      Caller MUST assume the plugin is not healthy. \\
    \bottomrule\addlinespace[1ex]
  \end{tabular}
  % \begin{tablenotes}
  % \end{tablenotes}
  \end{threeparttable}
    \caption{Probe gRPC Code Explained}
    \label{tab:probe-grpc-code-explained}
  \end{table}

\subsubsection{DeleteVolume}

A Controller Plugin MUST implement this RPC call if it has
\verb=CREATE_DELETE_VOLUME= capability. This RPC will be called by the CO to
deprovision a volume.

This operation MUST be idempotent. If a volume corresponding to the
specified \verb=volume_id= does not exist or the artifacts associated with the
volume do not exist anymore, the Plugin MUST reply 0 OK.

CSI plugins SHOULD treat volumes independent from their snapshots.

If the Controller Plugin supports deleting a volume without affecting
its existing snapshots, then these snapshots MUST still be fully
operational and acceptable as sources for new volumes as well as appear
on ListSnapshot calls once the volume has been deleted.

When a Controller Plugin does not support deleting a volume without
affecting its existing snapshots, then the volume MUST NOT be altered in
any way by the request and the operation must return the
\verb=FAILED_PRECONDITION= error code and MAY include meaningful human-readable
information in the status.message field.

\inputminted[breaklines,frame=lines,linenos,fontsize=\footnotesize]{proto}{src/deletevolume.proto}

{\bfseries{DeleteVolume Errors}} \\
If the plugin is unable to complete the DeleteVolume call successfully,
it MUST return a non-ok gRPC code in the gRPC status. If the conditions
defined below are encountered, the plugin MUST return the specified gRPC
error code. The CO MUST implement the specified error recovery behavior
when it encounters the gRPC error code.

  \begin{table}[H]
  \begin{threeparttable}
    \begin{tabular}{p{5em}p{11em}p{8em}p{8em}}
    \toprule
    \thead{\bfseries Condition} & \thead{\bfseries gRPC Code} & 
    \thead{\bfseries Description} & \thead{\bfseries Recovery Behavior} \\
    \midrule
      Volume in use & 9 FAILED\_PRECONDITION & Indicates that the
      volume corresponding to the specified volume\_id could not be
      deleted because it is in use by another resource or has snapshots
      and the plugin doesn't treat them as independent entities. &
      Caller SHOULD ensure that there are no other resources using the
      volume and that it has no snapshots, and then retry with
      exponential back off. \\
    \bottomrule\addlinespace[1ex]
  \end{tabular}
  % \begin{tablenotes}
  % \end{tablenotes}
  \end{threeparttable}
    \caption{DeleteVolume gRPC Code Explained}
    \label{tab:delete-volume-grpc-code-explained}
  \end{table}

\subsubsection{ControllerPublishVolume}
\subsubsection{ControllerUnpublishVolume}
\subsubsection{ListVolumes}
\subsubsection{ControllerGetVolume}
\subsubsection{GetCapacity}
\subsubsection{ControllerGetCapabilities}
\subsubsection{CreateSnapshot}
\subsubsection{DeleteSnapshot}
\subsubsection{ListSnapshots}
\subsubsection{ControllerExpandVolume}

\subsection{Node Service RPC}
\subsubsection{NodeStageVolume}
\subsubsection{NodeUnstageVolume}
\subsubsection{NodeUnPlubishVolume}
\subsubsection{NodeGetVolumeStats}
\subsubsection{NodeGetCapabilities}
\subsubsection{NodeGetInfo}
\subsubsection{NodeExpandVolume}

\newpage
\section{Protocol}
\subsection{Connectivity}
\subsection{Security}
\subsection{Debugging}

\newpage
\section{Configuration and Operation}
\subsection{General Configuration}
\subsubsection{Plugin Bootstrap Example}
\subsubsection{Filesystem}
\subsection{Supervised Lifecycle Management}
\subsubsection{Environment Variables}
\subsubsection{Operational Recommendations}

\end{document}
% vim: set ai nu nobk expandtab ts=4 sw=2 tw=72 syntax=tex :
